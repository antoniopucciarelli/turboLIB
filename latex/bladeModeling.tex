\subsection{Losses modeling} \label{lossesModeling}
\subsubsection{Profile losses}
	{\nologo
	\begin{frame}{Profile losses}
		The profile losses used are related to the \textbf{Leiblein modeling} approach\footnote{The following equations are interpolated data from \cite[Sec. 6.4]{axial2004}.}.
		\newline
		The model is based on the \textbf{equivalent diffusion factor}\footnote{It describes how important is the \textbf{velocity change} along the blade. It can be seen as an \textit{indicator} of the \textbf{blade loading}.}, $D_{eq}$:
		\begin{align}
			\frac{W_{max}}{W_1} & = 1.12 + 0.61 \ \frac{cos(\beta_1)^2}{\sigma} \cdot \frac{r_1 \ V_{t1} - r_2 \ V_{t2}}{r_1 \ V_{a1}} \nonumber \\
			D_{eq} & = \frac{W_{max}}{W_1} \cdot \frac{W_1}{W_2} \nonumber  
		\end{align}
		$D_{eq}$ will be used for the computation of $\bar{\omega}_{profile}$ as:
		\begin{equation}
			\bar{\omega}_{profile} = \frac{0.004 \ \Big( 1 + 3.1 \ \big( D_{eq} - 1 \big)^2 + 0.4 \ \big( D_{eq} - 1 \big)^8 \Big) \ 2 \ \sigma}{cos(\beta_2) \ \Big( \frac{W_1}{W_2} \Big)^2} \nonumber
		\end{equation}
	\end{frame}
	}

\subsubsection{Compressibility losses}
	{\nologo
	\begin{frame}{Compressibility losses -- \Romannum{1}}
		 These losses can be seen as a \textbf{correction} of the \textbf{profile} losses due to the compressibility of the gas along its \textit{journey} in the stage. 
		 \newline
		 \newline
		 The correction refers to a \textbf{Leiblein correction} model that uses the \textbf{positive} and \textbf{negative} blade section incidence angle\footnote{$i_c$ and $i_s$ are related to the total pressure losses, $\bar{\omega}_c$ and $\bar{\omega}_s$, that are \textbf{twice} the minimum total pressure loss, $\bar{\omega}$, obtained at the \textbf{design incidence angle}, $i^*$.}, $i_c$ and $i_s$. 
		 \newline
		 \newline
		 These new stall incidence angles will build a new \textbf{mean} incidence angle, $i_m$, that can be seen as the \textbf{optimum} incidence angle related to the inlet Mach conditions\footnote{The implemented model follows \cite[Sec. 6.6]{axial2004}}. 
	\end{frame}
	}
	\begin{frame}{Compressibility losses -- \Romannum{2}}
		$\bar{\omega}_{compressibility}$ setup:
		\begin{itemize}
			\item $R_c$ and $R_s$ computation\footnote{$R_{c}$ and $R_{s}$ are \textbf{range indices} for the computation of $i_c$ and $i_s$.}:
				\begin{align}
					R_c & = 9 - \Bigg[1 - \Bigg( \frac{30}{\beta_1} \Bigg)^{0.48} \Bigg] \ \frac{\theta}{8.2} \nonumber \\
					R_s & = 10.3 + \Bigg( 2.92 - \frac{\beta_1}{15.6} \Bigg) \ \frac{\theta}{8.2} \nonumber  
				\end{align}
			\item $i_c$ and $i_s$ computation due to \textbf{compressibility} effects:
				\begin{align}
					i_c & = i^{*} - \frac{R_c}{1 + 0.5 \ M_{1}^{3}} \nonumber \\
					i_s & = i^{*} + \frac{R_s}{1 + 0.5 \Big( K_{sh} \ M_{1} \Big)^3} \nonumber 
				\end{align}
		\end{itemize}
	\end{frame}

	{\nologo
	\begin{frame}{Compressibility losses -- \Romannum{3}}
		\begin{itemize}
			\item $i_m$ computation\footnote{$i_m$ is the incidence angle at wich corresponds, having accounted the \textbf{compressibility}, the \textbf{minimum pressure loss}, $\bar{\omega}_m$.}: $i_m = i_c + \Big( i_s - i_c \Big) \ \frac{R_c}{R_c + R_s}$
			\item $\bar{\omega}_{m}$ computation: $\bar{\omega}_{m} = \bar{\omega}_{profile} \ \Bigg[ 1 + \frac{\big( i_m - i^{*} \big)^2}{R_s^{2}} \Bigg]$
			\item $\bar{\omega}_{compressibility}$ computation\footnote{After having computed $i_c$, $i_s$, $i_m$ and $\bar{\omega}_m$. $\bar{\omega}_{compressibility}$ is a function of $i$. In the \textbf{total pressure losses study}: $\bar{\omega}_{compressibility} = \bar{\omega}_{compressibility_{(i^*)}}$.}: 
				\begin{align}
					\bar{\omega}_{compressibility} = \bar{\omega}_m + \bar{\omega}_m \ \Bigg[ \frac{i - i_m}{i_c - i_m} \Bigg]^2 \text{, if } i \leq i_m\nonumber \\ 
					\bar{\omega}_{compressibility} = \bar{\omega}_m + \bar{\omega}_m \ \Bigg[ \frac{i - i_m}{i_s - i_m} \Bigg]^2 \text{, if } i \geq i_m \nonumber 
				\end{align}
		\end{itemize}
	\end{frame}
	}

\subsubsection{Secondary flow losses}
	\begin{frame}{Secondary flow losses}
		These losses are relative to the \textbf{secondary flow} inside the compressor and are \textit{usually} \textbf{greater} than the other losses. These losses are related to \textbf{eddies} generated with the \textbf{blade-flow} interaction and \textbf{streamlines} displacement due to the presence of \textbf{pressure gradients}. 
		\newline 
		These losses are computed with \textbf{Howell}'s model \cite[Ch. 6]{axial2004}\footnote{Howell computed a secondary flow loss model that is automatically embedded into $\bar{\omega}_{profile}$: this model is used for the \textbf{estimation} of the stator/rotor \textbf{blade number}.}:
		\begin{align}
			\bar{\beta} & = \frac{arctan( tan( \beta_1 ) + tan( \beta_2 ) )}{2}; & \text{\Romannum{1}} \nonumber \\ 
			C_{L} & = 2 \ cos( \bar{\beta} ) \cdot \frac{tan( \beta_1 ) - tan( \beta_2 )}{\sigma}; & \text{\Romannum{2}} \nonumber \\
			C_{D} & = 0.18 \ C_{L}^2; & \text{\Romannum{3}} \nonumber \\
			\bar{\omega}_{secondary} & = C_{D} \ \sigma \cdot \frac{cos( \beta_1 )^2}{cos( \bar{\beta} )^3}; & \text{loss computation} \nonumber 
		\end{align}
	\end{frame}

\subsubsection{End wall losses}
	\begin{frame}{End wall losses}
		These losses are related the interaction between the flow and the \textbf{compressor case}. They are \textit{lower} than the \textbf{secondary flow} losses.
		\newline
		It has been used a simple and fast relation made by \textbf{Howell} \cite[Ch. 6]{axial2004}\footnote{This loss is kept into account in the \textbf{blade numbering} study.}:
		\begin{align}
			C_{D} & = 0.02 \ \frac{s}{b_{0}} \nonumber \\ 
			\bar{\omega}_{endWall} & = C_{D} \ \sigma \cdot \frac{cos( \beta_1 )^2}{cos( \bar{\beta} )^3}; & \text{loss computation} \nonumber 
		\end{align}		
	\end{frame}

\subsubsection{Shock losses}
	\begin{frame}{Shock losses -- \Romannum{1}}
		The \textbf{relative} Mach number at the rotor inlet is slightly above \textbf{sonic speed}; a \textbf{shock wave} will be present at the rotor tip. From \cite{manfredi2020transonic}, \textbf{shock pattern} is related to \textbf{Mach number} and \textbf{airfoil shape}.
		\newline
		\newline 
		The \textbf{shock} losses modeling is related to \textbf{K\"onig losses} modeling approach. This model describes a \textbf{2 shock waves loss} using a \textbf{single normal shock} with respect to a computed Mach number, $M_{in}$. 
		\newline
		\newline
		\textbf{K\"onig model} depends mainly on \textbf{blade deflection angle}, $\theta$, and \textbf{relative inlet Mach}\footnote{\textbf{Leading edge radius} is not taken into account due to the \\ \textit{approximated nature} of the model.}, $M_1$.
	\end{frame}
	
	\begin{frame}{Shock losses -- \Romannum{2}}
		\textbf{Swan} and \textbf{Miller}, \cite[Sec. 6.7]{axial2004}, derived a \textbf{shock loss formulation} from \textbf{K\"onig model}. 
		\newline 
		\newline
		The following are the steps made for the computation of the \textbf{shock loss}, $\bar{\omega}_{shock}$, at each \textbf{blade section}:
		\begin{itemize}
			\item computation of the \textbf{expansion wave} angle, $\phi$: 
				\begin{equation}
					\phi = \frac{s \ cos(\psi)}{s \ sin(\psi) \ R_u}; \text{ where } \psi = \psi_{(\beta_1, \ \gamma, \ \theta)} \nonumber
				\end{equation}
			\item computation of $W_s$ and $M_s$ using \textbf{Prandtl-Meyer} expansion: 
			\begin{equation}
				\phi = \int_{W_1}^{W_s} \sqrt{M^2 - 1} \ \frac{dW}{W} \nonumber
			\end{equation}
			\item $M_{in}$ computation: $M_{in} = \sqrt{M_1 \ M_s}$
			\item \textbf{normal shock} solution and computation of $\Delta P_T$
			\item from $\Delta P_T$, computation of $\bar{\omega}_{shock}$
		\end{itemize}
	\end{frame}

\subsubsection{Tip leackage losses}
	{\nologo
	\begin{frame}{Tip leackage losses}
		Again these losses are computed from \cite[Sec. 6.9]{axial2004}. The main concept is: computing a \textbf{total} blade pressure loss and \textbf{assuming linear distribution} of losses from the hub to the tip\footnote{$Z$ is the number of blades. $\delta_c$ is the tip clearance. $N_{row}$ is the number of blade rows in the compressor.}. 
		\begin{align}
			\tau & = \pi \ \delta_c \Big[ r_1 \rho_1 V_{a1_{mean}} + r_2 \rho_2 V_{a2_{mean}} \Big] \Big[ r_2 V_{t2_{mean}} - r_1 V_{t1_{mean}} \Big] \nonumber \\ 
			\Delta P & = \frac{\tau}{Z \ r_{tip} \delta_c \ c \ cos(\gamma)} \nonumber \\ 
			U_c & = 0.816 \frac{\sqrt{\frac{2 \Delta P}{\rho_{mean}}}}{N_{row}^{0.2}} \nonumber \\ 
			\dot{m_c} & = \rho_{mean} \ U_c \ Z \ \delta_c \ c \ cos(\gamma) \nonumber \\ 
			\Delta P_T & = \frac{\Delta P \ \dot{m_c}}{\dot{m}} \nonumber  
		\end{align}
		$\Delta P_T$ is the \textbf{overall} total pressure loss due to \textbf{tip leackage}.
		\vspace{0.11cm}
	\end{frame}
	}

\subsection{Blade shape}
	{\nologo
	\begin{frame}{Blade shape setup}
		The \textbf{Leiblein model} from \cite[Ch. 6]{axial2004} has been used for the blade shape computation. 
		\newline 
		\newline
		$\mathtt{NACA-65}$ profile has been chosen for the blade generation\footnote{Due to the low tip sonic Mach number it has been chosen to use this profile as well for the blade tip instead of a supersonic adapted profile shape.}.
		\newline
		\newline
		The \textbf{main contraints} are: $\frac{t_b}{c} \approx 0.1$ and $\sigma$ is related just to $s$\footnote{$\frac{t_b}{c} \approx 0.1$ allows setting $K_{sh} \approx 0.1$. The blade chord, $c$, is set up as constant during the \textbf{blade numbering} study: using $AR = \frac{b_0}{c}$.}.
		\newline
		\newline
		Due to the many possible blade configurations, an \textbf{optimization} procedure has been used for the computation of $i^{*}$, $\delta$ and $\theta$. 
		\newline
		\newline
		Each section airfoil shape is \textbf{computed} from $\theta$ and $\mathtt{NACA-65} \ C_{L0}$ surface coordinates.
	\end{frame}
	}

	\begin{frame}{$i^*$ computation}
		The \textbf{incidence angle}, $i^*$, is computed using:
		\begin{align}
			K_{t,i} & = \Bigg( 10 \ \frac{t_b}{c} \Bigg)^q \ \text{; where} \ q = \frac{0.28}{0.1 + (\frac{t_b}{c})^{0.3}} \nonumber \\ 
			(i^*_0)_{10} & = \frac{\beta_0^p}{5 + 46 \cdot e^{-2.3 \sigma}} - 0.1 \ \sigma^{3} \ e^{\frac{\beta_0 - 70}{4}} \ \text{; where} \ p = 0.914 + \frac{\sigma^3}{160} \nonumber \\
			n & = 0.025 \sigma - 0.06 - \frac{\Big(\frac{\beta_0}{90} \Big)^{1 + 1.2 \sigma}}{1.5 + 0.43 \sigma} \nonumber \\ 
			i^* & = K_{sh} \ K_{t,i} \ (i^*_0)_{10} + n \ \theta \nonumber 
		\end{align}
	\end{frame}

	\begin{frame}{$\delta$ computation}
		The \textbf{deviation angle}, $\delta$, is computed using:
		\begin{align}
			K_{t,\delta} & = 6.25 \frac{t_b}{c} + 37.5 \Bigg(\frac{t_b}{c}\Bigg)^2 \nonumber \\ 
			(\delta^*_0)_{10} & = 0.01 \sigma \beta_0 + (0.74 \sigma^{1.9} + 3 \sigma) \Bigg(\frac{\beta_0}{90}\Bigg)^{1.67 + 1.09 \sigma} \nonumber \\
			b & = 0.9625 - 0.17 \frac{\beta_0}{100} - 0.85 \Bigg(\frac{\beta_0}{100}\Bigg)^3 \nonumber \\ 
			m & = \frac{m_{1.0}}{\sigma^b} \ \text{; where} \ m_{1.0} = 0.17 - 0.0333 \frac{\beta_0}{100} + 0.333 \Bigg(\frac{\beta_0}{100}\Bigg)^2 \nonumber \\
			\delta & = K_{sh} \ K_{t,\delta} \ (\delta^*_0)_{10} + m \ \theta \nonumber  
		\end{align}
	\end{frame}

	\begin{frame}{$\theta$ computation}
		Starting from the \textbf{known} flow deflection angle\footnote{$\varepsilon = \beta_{inlet} - \beta_{outlet}$. $\beta_1$ and $\beta_2$ change with respect to the \textbf{axial speed}.}, $\varepsilon$, it is necessary to compute:
		\begin{equation}
			\theta = \varepsilon - i^* + \delta \nonumber
		\end{equation}
		Since $i^*$ and $\delta$ are functions of $\theta$, the computation of $\theta$ is made by an \textbf{iterative process}:
		\begin{equation}
			\theta = \varepsilon - i^*_{(\theta)} + \delta_{(\theta)} \nonumber
		\end{equation}
		Once found $\theta$, the \textbf{total pressure loss} coefficients, $\bar{\omega}_{*}$, are computed.   
	\end{frame}

\subsection{Radial equilibrium}
	\begin{frame}{Equation setup}
		The \textbf{radial equilibrium} equation: $\frac{\partial h_t}{\partial r} = T \frac{\partial s}{\partial r} + V_a \frac{\partial V_a}{\partial r} + V_t \frac{\partial r V_t}{\partial r}$ is converted into, for the exit station\footnote{$1$ is the blade inlet station and $2$ is the blade outlet section.} of the blade, a \textbf{1st order ODE}:
		\begin{equation}
			\begin{split}
				- \frac{1}{2} \frac{\partial V_{a2}^2}{\partial r} + \frac{V_{a2}^2}{2 \ c_P} \frac{\partial s_{2}}{\partial r} = - c_P \frac{\partial T_{T1}}{\partial r} - \omega \ \frac{\partial r V_{t2}}{\partial r} + \omega \ \frac{\partial r V_{t1}}{\partial r} \\ + T_{T1} \ \frac{\partial s_2}{\partial r} + \frac{\omega}{c_P} r V_{t2} \ \frac{\partial s_2}{\partial r} - \frac{\omega}{c_P} r V_{t1} \frac{\partial s_2}{\partial r} - \frac{1}{2 \ c_P} V_{t2}^2 \frac{\partial s_2}{\partial r} + \frac{V_{t2}}{r} \frac{\partial r V_{t2}}{\partial r}	
			\nonumber
			\end{split}
		\end{equation}
		\begin{itemize}
			\item The \textbf{ODE} will be solved for $V_{a2}^2$ 
			\item $s_2$ is computed from\footnote{$\bar{\omega}_{*}$ computation has been treated earlier in \hyperlink{lossesModeling}{{\color{blue} \nameref{lossesModeling}}}.} $\sum_i \bar{\omega}_{i}$.  
			\item $\omega, V_{t1}, V_{t2} \ \text{and} \ T_{T1}$ are known
		\end{itemize}
	\end{frame}
	
	\begin{frame}{$\Delta s$ computation}
		From pressure loss coefficients it is possible compute the \textbf{outlet total pressure} as: 
		\begin{equation}
			P_{T2,r} = P_{T1,r} + \sum_i \bar{\omega}_{i} \ (P_{T1,r} - P_{1,r}) \nonumber
		\end{equation}
		The \textbf{entropy variation} is computed as: 
		\begin{equation}
			\Delta s = s_2 - s_1 = c_P \log{\frac{T_{T2,r}}{T_{T1,r}}} - R \log{\frac{P_{T2,r}}{P_{T1,r}}} \nonumber
		\end{equation}
		\vspace{-0.5cm}
		\begin{alertblock}{Frames}
			\begin{itemize}
				\item $T_{T2} = T_{T1}$ in \textbf{stators} and $T_{T2,r} = T_{T1,r}$ in \textbf{rotors}
				\item $\bar{\omega}_{*}$ have to be computed using \textbf{relative} quantities for \textbf{rotors} and \textbf{absolute} quantities for \textbf{stators}
				\item $T_{T,r} = T + \frac{W^2}{2 c_P} = T_T + \frac{W^2 - V^2}{2 c_P}$
				\item $P_{T,r} = P_T \Big( \frac{T_{T,r}}{T_T} \Big)^{\frac{\gamma}{\gamma - 1}}$
			\end{itemize}
		\end{alertblock}
	\end{frame}
